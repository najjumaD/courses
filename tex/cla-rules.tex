
%\twocolumn

\section*{Exam appendix --- Rules of Algebra}
%Let $a$, $b$, $c$, and $d$ be any real numbers. \pbn

\subsubsection*{Basic Arithmetic}
\begin{minipage}{0.5\textwidth}
	\abcx{\setlength{\itemsep}{6pt}
		\item $a + b = b + a$
		\item 	$( a+b ) +c = a+ ( b+c )$
		\item $	a + 0 = a$
		\item $	a + ( -a ) = 0$
		\item $	ab = ba$
		\item $	( ab ) c = a ( bc )$
		\item $1\cdot a = a$
		\item $aa^{-1} = 1$, for $a\neq 0$
		\item $	(- a ) b = a ( -b ) = -ab$
		\item $	( -a )( -b ) = ab$
		\item $	a ( b + c ) = ab + ac$
		\item $	( a + b ) c = ac + bc$
		%\itex{
		\item $( a + b )^2 = a^2 + 2ab + b^2$
		\item $( a -b )^2 = a^2- 2ab + b^2$
		\item $( a + b ) ( a -b ) = a^2- b^2 $
	}
\end{minipage}
\begin{minipage}{0.5\textwidth}
	
	\subsubsection*{Rules for Fractions}
	\abcx{\setlength{\itemsep}{6pt}
		\item $\frac{a}{b}+\frac{a}{d}=\frac{ad+bc}{bd}$
		\item $\frac{a}{b}\cdot \frac{c}{d} = \frac{ac}{bd}$
		\item $\frac{a}{b}:\frac{c}{d}=\frac{\frac{a}{b}}{\frac{c}{d}}=\frac{a}{b}\cdot \frac{d}{c}=\frac{ad}{bc}$
		\item $\frac{a}{b}\cdot \frac{c}{c}=\frac{ac}{bc}$
	}
					\end{minipage}
	\begin{minipage}{0.5\textwidth}
	\subsubsection*{Rules for Powers}
	\abcx{\setlength{\itemsep}{6pt}
		\item $x^0=1$
		\item $x^a\cdot x^b=x^{a+b}$
		\item $\frac{x^a}{x^b}=x^{a-b}$
		\item $x^a\cdot y^a=(x\cdot y)^a$
		\item $\left(\frac{x}{y}\right)^a=\frac{x^a}{y^a}$
		\item $(x^a)^b=x^{a\cdot b}$
		\item $x^a\cdot y^b=z^c \Leftrightarrow x^{\frac{a}{c}}\cdot y^{\frac{b}{c}}=z$
	}
\end{minipage}
\begin{minipage}{0.5\textwidth}

\subsubsection*{Rules for Roots}

\abcx{\setlength{\itemsep}{6pt}
	\item $a^{\frac{1}{2}}=\sqrt{a}$ 
	\item $\sqrt[n]{a}=a^{\frac{1}{n}}$ (valid if $a\geq 0$) 
	\item $a^{\frac{m}{n}}=\sqrt[n]{a^m}$
	\item $\frac{1}{\sqrt[n]{a^m}}=a^{-\frac{m}{n}}$
	\item $\sqrt{ab}=\sqrt{a}\sqrt{b}$
	\item $\sqrt{a+b}\neq \sqrt{a}+\sqrt{b}$	
	\item $\sqrt{\frac{a}{b}}=\frac{\sqrt{a}}{\sqrt{b}}$
}
\end{minipage}
%\begin{minipage}{0.5\textwidth}

\subsubsection*{Rules for Quadratic Function}

\itex{
	\item The \textbf{roots} of a quadratic function, i.e. $ax^2+bx+c=0$, are defined by the formula $$x_{1,2}= \frac{-b\pm\sqrt{b^2-4ac}}{2a}$$ if  $b^2 - 4ac \geq 0$.
	\item Factorization of a  quadratic function also gives us the roots:$$ax^2+bx+c=a(x-x_1)(x-x_2)$$
%	\textbf{Example:} $$G(x)=-3x^2+90x-375= -3(x-5)(x-25)$$
}
%\end{minipage}

%\onecolumn

\pbn
\subsubsection*{Rules of Differentiation}

\begin{align*}
	\text{Rule 1:}\quad	f(x)&=A \\
	\Rightarrow  f^{\prime}(x)&=0\\
	\text{Rule 2:}\quad	f(x)&=A+g(x)  \\
	\Rightarrow  f^{\prime}(x)&=g^{\prime}(x)\\
	\text{Rule 3:}\quad	f(x)&=Ag(x)  \\
	\Rightarrow  f^{\prime}(x)&=A g^{\prime}(x)\\
	\text{Rule 4 (power):}\quad	f(x)&=x^{a} \\
	\Rightarrow  f^{\prime}(x)&=a x^{a-1}
\end{align*}
with a being an arbitrary constant.

If both $f$ and $g$ are differentiable at $x,$ then the sum $f+g$ and the difference $f-g$ are both differentiable at $x,$ and
\begin{align*}
	\text{Rule 5 (sums):}\quad	h(x)&=f(x) \pm g(x)\\
	\Rightarrow \quad h^{\prime}(x)&=f^{\prime}(x) \pm g^{\prime}(x)
\end{align*}

\pbn
If both $f$ and $g$ are differentiable at $x$, then so is $h=f \cdot g$,  and 
\begin{align*}
	\text{Rule 6 (products):}\quad h(x)&=f(x) \cdot g(x) \\
	\Rightarrow \quad h^{\prime}(x)&=f^{\prime}(x) \cdot g(x)+f(x) \cdot g^{\prime}(x)
\end{align*}



\pbn
If both $f$ and $g$ are differentiable at $x$ and $g(x) \neq 0,$ then $h=f / g$ is differentiable at $x,$ and 
\begin{align*}
	\text{Rule 7 (quotient):}\quad	h(x)&=\frac{f(x)}{g(x)} \\
	\Rightarrow \quad h^{\prime}(x)&=\frac{f^{\prime}(x) \cdot g(x)-f(x) \cdot g^{\prime}(x)}{(g(x))^{2}}
\end{align*}



\pbn
If $g$ is differentiable at $x$ and $f$ is differentiable at $u=g(x),$ then the composite function $h(x)=f(g(x))$ is differentiable at $x,$ and

\begin{align*}
	\text{Rule 8 (sums):}\quad	h^{\prime}(x)&=f^{\prime}(u) \cdot g^{\prime}(x)=f^{\prime}(g(x)) \cdot g^{\prime}(x)
\end{align*}

\textbf{In words:} First differentiate the exterior function with respect to the interior function (kernel), then multiply by the derivative of the interior function.






\pbn
\subsubsection*{Rules for Summation}
The \textbf{summation sign}, $\sum$, allows for a compact formulation of
lengthy expressions.


Rule  (Additivity Property)
$$
\sum_{i=1}^{n}\left(a_{i}+b_{i}\right)=\sum_{i=1}^{n} a_{i}+\sum_{i=1}^{n} b_{i}
$$
Rule  (Homogeneity Property)
$$
\sum_{i=1}^{n} c a_{i}=c \sum_{i=1}^{n} a_{i} \\
$$
and if $a_{i}=1$ for all $i$ then
$$
\sum_{i=1}^{n} c a_{i}=c \sum_{i=1}^{n} a_{i}=c(n \cdot 1)=c n
$$
Rule (Rule for Sums)
\begin{align*}
	\sum_{i=1}^{n} i&=1+2+\ldots+n=\frac{1}{2} n(n+1) \\
	\sum_{i=1}^{n} i^{2}&=1^{2}+2^{2}+\ldots+n^{2}=\frac{1}{6} n(n+1)(2 n+1) \\
	\sum_{i=1}^{n} i^{3}&=1^{3}+2^{3}+\ldots+n^{3}=\left(\frac{1}{2} n(n+1)\right)^{2}=\left(\sum_{i=1}^{n} i\right)^{2}
\end{align*}
$$
\sum_{i=0}^{n} a^{i}=\frac{1-a^{n+1}}{1-a}
$$

The double sum notation allows us to write lengthy expressions in a compact way.
$$
\sum_{i=1}^{Z} b_{i} \sum_{j=1}^{S} a_{i j} b_{j}=\sum_{i=1}^{Z} \sum_{j=1}^{S} a_{i j} b_{i} b_{j}=\sum_{j=1}^{S} \sum_{i=1}^{Z} a_{i j} b_{i} b_{j}=\sum_{j=1}^{S} b_{j} \sum_{i=1}^{Z} a_{i j} b_{i}
$$
Consider some summation sign $\sum_{i=1}^{Z} .$ All variables with index $i$ must be to the right of that summation sign.


\subsubsection*{Rules for Logarithms and Exponentials}

\subsubsection*{Exponential Function}
\begin{align*}
	y&=f(x)=e^x=exp(x) \quad \text{is strictly increasing } \forall x \in \mathbb R \\
	D_f&=\mathbb R= ]-\infty,\infty[\\
	e^x&> 0 \forall x \in \mathbb R\\
	e^x&= 1+x+\frac{x^2}{2}+\frac{x^3}{4}+\frac{x^4}{24}+\dots\\
	e^0&= 1\\
	e^1&= e= 2.718\dots \text{Euler number}\\
	\lim\limits_{x\rightarrow\infty}e^x&\rightarrow\infty \quad \text{and}\\
	\quad \lim\limits_{x\rightarrow - \infty}e^x& \rightarrow 0
\end{align*}

\begin{minipage}{0.5\textwidth}
\subsubsection*{Logarithmic Function}
\begin{align*}
	y&=f(x)=\ln(x) \quad \text{is strictly increasing } \\
	D_f&=\mathbb R^+=]0,\infty[\\
	\ln(x)&=e^{\ln x}\\
	\ln 1&= 0\\
	\ln e&= 1
\end{align*}
\end{minipage}
\begin{minipage}{0.5\textwidth}
	
\subsubsection*{Rules to Transform exp}
\begin{align*}
	e^{x+y}&=e^x\cdot e^y\\
	e^{x-y}&=\frac{e^x}{e^y}\\
	e^{ax}&=(e^x)^a\\
	e^{\ln x}&=x\\
\end{align*}
\end{minipage}
\begin{minipage}{0.5\textwidth}
	
\subsubsection*{Rules to Transform log}
\begin{align*}
	\ln(x\cdot y)&=\ln x + \ln y\\
	\ln(\frac{x}{y})&=\ln x - \ln y\\
	\ln x^a&=a\cdot \ln x\\
	\ln e^x&= x
\end{align*}
\end{minipage}
\begin{minipage}{0.5\textwidth}
	
\subsubsection*{Rules for Derivatives of Both Functions}
\begin{align*}
	f(x)&=e^x \Rightarrow f'(x)=e^x\\
	f(x)&=\ln x \Rightarrow f'(x)=\frac{1}{x}\\
	\left(e^{f(x)}\right)^{\prime}&=f^{\prime}(x) \cdot e^{f(x)}\\
	(\ln f(x))^{\prime}&=f^{\prime}(x) \cdot \frac{1}{f(x)}
\end{align*}
\end{minipage}





