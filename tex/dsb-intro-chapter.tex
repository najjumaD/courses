


\chapter{Introduction: What is data science and data analysis?}
\includegraphics[width=.6\textwidth]{$HOME/Dropbox/hsf/pic/jdm/automated}

\section{Five books to learn data science and data analysis}

\includegraphics[width=0.19\textwidth]{../../../pic/dskell}
\includegraphics[width=0.2\textwidth]{../../../pic/bekes}
\includegraphics[width=0.18\textwidth]{../../../pic/hartford}
\includegraphics[width=0.2\textwidth]{../../../pic/nhk}
\includegraphics[width=0.18\textwidth]{../../../pic/mixtape}

\itex{
	\item \bibentry{Kelleher2018Dataa} \citep{Kelleher2018Dataa}
	\item \bibentry{Harford2020How} \citep{Harford2020How}
	\item \bibentry{Bekes2021Data} \citep{Bekes2021Data}
	\item \bibentry{Huntington-Klein2021Effect}\footnote{Freely available online: \url{https://theeffectbook.net}} \citep{Huntington-Klein2021Effect}
	\item \bibentry{Cunningham2021Causal}\footnote{Freely available online: \url{https://mixtape.scunning.com}} \citep{Cunningham2021Causal}
}


\subsection*{Data science is what data scientists do}


\textbf{Data analysts\dots}
\itex{\item \dots collect, manipulate, visualize, and analyze data. 
	\item \dots aim to discover useful information to support decision-making.
	\item \dots should help to understand and optimize processes and to open up new product areas and/or lines of business.
	\item \dots need to interact with customers, i.e., the user of the information as the addressee, and with domain experts.}

\textbf{Data analysts should have\dots}
\itex{
	\item \dots econometric skills (Regressions, Time Series Analysis, Machine Learning)
	\item \dots knowledge on statistical software packages (R, Phyton, SAS, etc.)
	\item \dots a basic understanding of business operations (Business Intelligence)
	\item \dots technical knowledge in database design, data models, data mining and segmentation techniques
}




\pbn
\section{My first PC}
\zitat{
	``Data science is best understood as partnership between a data scientist and a computer.''
	\hfill --- \citet[p. 97]{Kelleher2018Data}
}

\begin{minipage}{0.29\linewidth}
Commodore C64, price  about \textbf{750 Euro}, CPU < 1 MHz, RAM 64 KByte, i.e., 0,064 MB or 0,000064 GB, Graphics 320 x 200 with 16 colors, no HD but 5,25" floppy disks of 166KByte and a max of 144 files.\\ My main PC today has 4100Mhz at 8 cores and 16 threats with 32GB of RAM and my video card can do 3840 x 2160 at four screens, 1 NVmE of 1 TB, 2 SSD of 500GB each, 2 HD of 4TB each and 3TB in a cloud.

My first external HD in 2002 had 160GB and cost 170 Euro.
	\end{minipage}	
\begin{minipage}{0.7\linewidth}
\begin{center}
	\includegraphics[width=0.9\linewidth]{../../../pic/C64c_system}
\end{center}
	\end{minipage}	

\exex{What is the house of Santa Claus?}{
	\begin{minipage}{0.7\linewidth}		
		The house of Santa Claus is an old German drawing game. It goes like this: You have to draw a house in one line where you (a) must start at bottom left (point 1), (b) you are not allowed to lift your pencil while drawing and (c) it is forbidden to repeat a line. During drawing you say: ``Das ist das Haus des Nikolaus''.\\
		
		How many ways to fail and succeed, respectively do exist? Can you think of a logical procedure that gives you a complete solution?
	\end{minipage}	
	\begin{minipage}{0.29\linewidth}	
		\includegraphics[width=0.8\linewidth]{../../../pic/jdm/nikolaus}
	\end{minipage}	

}

\pbn
\solx{What is the house of Santa Claus?}{
	\begin{minipage}{0.4\linewidth}		
		\includegraphics[width=0.4\linewidth]{../../../pic/jdm/nikolaus2}
		\includegraphics[width=0.5\linewidth]{../../../pic/jdm/nikolaus3}
	\end{minipage}	
	\begin{minipage}{0.6\linewidth}	
		\websmall \url{https://de.wikipedia.org/wiki/Haus_vom_Nikolaus}\\
		There are 44 solutions and only 10 different ways to fail. Thus, the probability to fail is about 18.5\% and hence the probability to succeed is about 81.5\%.
		In the course 10 persons participated in the poll. Here are the answers:\\
		\includegraphics[width=0.8\linewidth]{../../../pic/jdm/nikolaus-poll}\\
		Nobody came close to the correct probability.		
	\end{minipage}	
}

\exex{Hexapawn}{	
\begin{minipage}{0.2\linewidth}	
	\includegraphics[width=.9\linewidth]{$HOME/Dropbox/hsf/pic/dsb/Hexapawn}
		
\end{minipage}	
\begin{minipage}{0.8\linewidth}	
\textbf{Hexapawn} is a simple game with the following rules:
As in chess, each pawn may be moved in two different ways: it may be moved one square forward, or it may capture a pawn one square diagonally ahead of it. A pawn may not be moved forward if there is a pawn in the next square. Unlike chess, the first move of a pawn may not advance it by two spaces. A player loses if they have no legal moves or the other player reaches the end of the board with a pawn. 
\abcx{
\item Play that game with a fellow student.
\item Think hard: Is there a way how you could improve your winning rate?
}
	\end{minipage}	
}

\pbn
\solx{Hexapawn}{
\begin{minipage}{0.5\linewidth}	
In his article \textbf{How to build a game-learning machine and then teach it to play and to win} \cite{Gardner1962How} discussed how a computer could be taught to play the game Hexapawn using a relatively small number of training matches.  The basic idea was to keep track of the different possible states of the board and the potential for success (i.e., a win) from each state. When a particular move led directly to a loss, the computer \textit{forgot} the move, thereby causing it  to  avoid  that  particular  loss  in  the  future.   By  \textit{pruning} possible  moves  in  this  way,  various intermediate game moves could indirectly lead to losses (i.e., to states that previously resulted in losses), and thus those intermediate moves would be pruned out as well.  This idea is a very simple version of \textit{reinforcement learning} which serves as the basis for today's machine learning gaming systems such as Google's Alpha-Go.
Gardner's original computer was constructed from matchboxes containing colored beads (this was the 1960's,  remember).  Each bead corresponded to a potential move,  and pruning involved disposing of the last bead played.  
\end{minipage}	
\begin{minipage}{0.49\linewidth}	
%	\includegraphics[width=.4\linewidth]{$HOME/Dropbox/hsf/pic/dsb/gardner_boxes}
		\includegraphics[width=.99\linewidth]{$HOME/Dropbox/hsf/pic/dsb/matchboxes}

To understand how the game-learning machine works, watch \tv \textit{The Game That Learns} \url{https://youtu.be/sw7UAZNgGg8} or watch \tv \textit{‘Building a MENACE machine’, Matthew Scroggs, University College London} \url{https://youtu.be/hK25eXRaBdc}
\end{minipage}	
}
%\solx{What is the house of Santa Claus?}{
%
%%
%}

\subsection*{Programming skills}
\begin{center}
	\includegraphics[width=.45\linewidth]{$HOME/Dropbox/hsf/pic/indeedjobs_png}
	\includegraphics[page=1, scale=0.2]{$HOME/Dropbox/hsf/lit/R_learn/Grolemund2018R.pdf}
\end{center}\note{Source: \cite{Muenchen2019Data}}


\subsection*{Data science skills are required in many jobs}
\begin{center}
	\includegraphics[width=.5\linewidth]{$HOME/Dropbox/hsf/pic/postings_pdf}
	\includegraphics[page=1, scale=0.2]{$HOME/Dropbox/hsf/lit/R_learn/Provost2013Data.pdf}
\end{center}	\nocite{Provost2013Data} \nocite{Grolemund2018R}


%https://newrelic-wpengine.netdna-ssl.com/wp-content/uploads/shutterstock_155746922_DataNerd.jpg

\subsection*{`Data scientist' is a weakly defined label}
\textbf{Business analysts\dots}
\itex{
	\item \dots also focus on data. 
	\item \dots rarely use sophisticated methods like machine learning or tools to analyse unstructured data
	\item \dots analyze data and assesses requirements from a business perspective related to an organization's overall system. 
	\item \dots are more concerned with the business implications of the data and the actions that should result (investment A vs. B). 
	\item \dots should leverage the work of data science teams to communicate an answer.}


\pbn
\subsection*{Stereotypes}

\begin{minipage}[t]{0.49\textwidth}
	\includegraphics[width=\textwidth]{$HOME/Dropbox/hsf/pic/nerdds_pdf}
	
	data scientists are nerds
\end{minipage}\hfill
\begin{minipage}[t]{0.49\textwidth}
	\includegraphics[width=\textwidth]{$HOME/Dropbox/hsf/pic/nerdbus_pdf}
	
	business analysts are networking moneymaker
\end{minipage}


See \websmall \url{http://www.worldofanalytics.be/blog/are-all-data-scientists-nerds} for a funny investigation of this topic.



\pbn
\subsection*{Data Science Approach: from prototype to production}
The goal is to use data in a way which creates value for a business.
\begin{center}
	\includegraphics[width=.6\linewidth]{$HOME/Dropbox/hsf/pic/one1_1}
\end{center}

\note{Source: \url{https://www.onelogic.de/en/data-science-services}}\includegraphics[width=.1\linewidth]{$HOME/Dropbox/hsf/pic/onelogic1}

\pbn
\subsection*{This lecture}
The goal is to use data in a way which creates value for a business.
\begin{center}
	\includegraphics[width=.6\linewidth]{$HOME/Dropbox/hsf/pic/one1_2}
\end{center}

\note{Source: \url{https://www.onelogic.de/en/data-science-services}}
\includegraphics[width=.1\linewidth]{$HOME/Dropbox/hsf/pic/onelogic1}


\section{Example: I am too lazy to read the book}
\begin{center}
	\citet{Provost2013Data}
	
	\includegraphics[page=1, scale=0.45,clip]{/home/sthu/Dropbox/hsf/lit/R_learn/Provost2013Data.pdf}
\end{center}

\subsubsection{My view as a data scientist on Ch. 1\&2 of \citet{Provost2013Data}}

\begin{center}
	\includegraphics[width=.5\linewidth]{$HOME/Dropbox/hsf/pic/acloud_ch1+2}\note{Source: own calculations}
\end{center}

The size of the word represents its frequency in the text, e.g., data was counted about 600 times, mining about 140 times and science about 100 times.

This is a \textbf{wordcloud}. A nice tool to analyze unstructured data such as the text of two chapters. Unstructured data does not have a pre-defined data model or is not organized in a pre-defined manner.

\section{Example: Taxi driver in NY}
\subsection{Best and worst job}
	\begin{center}
\includegraphics[width=.56\linewidth]{$HOME/Dropbox/hsf/pic/ds2}\\
\includegraphics[width=.56\linewidth]{$HOME/Dropbox/hsf/pic/td}
	\end{center}
University Professor is ranked third with the best work environment and less stress, but only a median salary of \$76,000.
\note{Source: https://www.careercast.com/jobs-rated/2019-jobs-rated-report}


\pbn
\subsection*{Can data science make taxi drivers happier?}

\paragraph{NYC Taxis: A Day in the Life}
\itex{\item Visit: \websmall \url{http://chriswhong.github.io/nyctaxi/}
\item This visualization displays the data for one random NYC yellow taxi on a single day in 2013. See where it operated, how much money it made, and how busy it was over 24 hours.
}
\boxb{\textbf{Get the data ready} Create a spreadsheet/dataset that contains all the shown information excluding the particular route taken by the taxi driver.
}

\pbn
\subsubsection{NYC Taxis: A Day in the Life}

\makebox[\linewidth]{\includegraphics[page=1,width=.7\paperwidth]{$HOME/Dropbox/hsf/pic/nyctaxi1}}
%		\includegraphics[width=1.1\linewidth]{../pic/nyctaxi1}

\pbn
\subsubsection{How to create and organize data}
\boxb{\textbf{Identifying Variables} dr\_id, date, op\_id, trip
}
\boxb{\textbf{What is the `Operation ID' and `trip'?} A Taxi can either be full or empty. With each change of this condition the ID must change. For example: A taxi starts empty (ID=1), then, the first passenger is picked up (ID=2), after that, a new trip begins when the taxi is empty again, \dots 
}
\boxb{\textbf{Attributes (Informing Variables)} Start-Coordinates (latitude/longitude), End-Coordinates (latitude/longitude), Start-Time, End-Time, Travel-Distance (miles), Fares, Surcharege, MTA Tax, Tips, Tolls, Number of Passengers
}


\subsubsection{Excerpt of the (structured) data:}
%\begin{table}[]
\begin{tabularx}{\linewidth}{llCCllCCCC}\toprule
dr\_id & date       & op\_id & trip &GPS\_1             & GPS\_2               & time\_1 & time\_2 & dist & fares  \\\midrule
A          & 03-15 & 1  &      1     & 40.7127837 -74.0059414 & {40.7137837 -74.0168515} & 09:00       & 09:10     & 1500     & 9                \\\hline
A          & 03-15 & 2 &        1    & 40.7137837 -74.0168515 & 40.7149827 -74.0157314 & 09:10       & 09:12     & 200      & 19                   \\\hline
A          & 03-15 & 1 &    2        & 40.7149827 -74.0157314 & 40.9137837 -74.0168517 & 09:12       & 09:55     & 35000    & 19 \\\hline
\dots         & \dots& \dots&\dots         & \dots&\dots         & \dots&\dots  &\dots  &\dots     \\\bottomrule
\end{tabularx}

\textbf{Please note:} This exemplary table is incomplete. In particular, the route of the trip is excluded. Real time travel information has a high granularity, i.e., tracking the taxi over time is more data intensive and the analysis is a real challenge.

\pbn
\subsubsection*{Are these sort of data `big data'?} 
For example, the T-Drive trajectory dataset of \cite{Zheng2011T} contains a one-week trajectories of 10,357 taxis. The total number of points in this dataset is about 15 million and the total distance of the trajectories reaches 9 million kilometers.

\pbn
\exex{What can data scientists do for taxi drivers?}{
Maybe a driver wants
\itex{
\item more profit (fare-costs+tips)
\item more tips 
\item less time being `empty' and driving around searching 
\item to know fuel saving-routes 
\item to work in areas with less stressful traffic (How to avoid Manhattan?)
\item to start at the optimal place and time 
\item to know when and where to rest without loosing much
\item ...

Discuss how data scientist may help.
}
}

\pbn
\exex{Helping Luigi}{
Suppose you are consulting the delivery service of Luigi's Pizza restaurant. So far, Luigi’s advertised his service in local television. Now, he commissioned you to use his database for targeted advertising. In particular, he wants find one district to start a poster advertising campaign.

Luigi gives you a large dataset with information on all 13,810 deliveries of the year 2018. Here is an excerpt of the first seven orders:

\includegraphics[width=.8\linewidth]{$HOME/Dropbox/hsf/pic/dsb/luigi_data}


Legend:\\
Order: Number of order (from 1 to 13,810).\\
Time: Time of order (24-hour clock).\\
Meal: Component of the orders. Overall, Luigi offers 100 meals (from 1 to 100).\\
Price: Total Price of order measured in Euro. Luigi does not charge delivery costs.\\
Address: Street name and number. All addresses are in the same city.\\
District: The city consists of four districts (A, B, C, D). All have about the same size.\\
Time to deliver: The time measured in minutes that it takes to drive the food to the customer.\bigskip


\abcx{
	\item Explain the structure of the dataset. In particular, name the variables that identify the dataset.
\item Suppose you are a data scientist. Name the services you can provide to Luigi.
\item You know little about the effectiveness of poster advertising and Luigi’s goals. Thus, in order to make the poster advertising campaign a success. You probably need more information. Luigi is well informed and open for your questions. Name four reasonable and concrete questions that you like to ask Luigi.
\item Luigi heart about the wordclouds and he thinks it is a modern tool that can help him. Explain Luigi what a wordcloud actually is and comment if it could really help him, or not.
}
}

\pbn
\solx{Helping Luigi}{
	\abcx{
\item This is a structured dataset. The identifying variable is \textbf{Order}.	
\item I can manipulate, visualize, and analyze the dataset for him. If necessary, I can collect and merge additional datasets to complement his information. Doing so, I can discover useful information to support his decision making. At best, I can help to understand and optimize his processes and to open up new product areas and/or lines of business. 
\item 
Possible answer:
\itex{
	\item What do you like to improve and maximize, respectively.
\item  Do you like to maximize volume of sales?
\item  Do you like to maximize the profit?
\item  If you like to maximize the profit, can you give information about the profit/contribution margin for each meal?
\item  What are the average costs for a driver per minute?
\item  Can you give me your address? I need it to calculate the travel distances between your restaurant and the address of the customer.
\item Do you think that a poster campaign help to attract new customers or increase the order frequency?
}
\item A wordcloud is a tool to visualize unstructured dataset. Mostly, it is used to identify the importance of specific words in a text. Luigi’s dataset only contains numeric variables with the addresses being the exception. Although a wordcloud could be used to visualize the popularity of meals or the frequency of orderings from particular districts, for example, it is probably better to do that in tables that show the frequency of occurrence for each meal and the number of orders by district.
}
}

\exex{AlphaGo}{
Watch \tv \textit{AlphaGo - The Movie | Full award-winning documentary} \url{https://youtu.be/WXuK6gekU1Y}
}





\section{What is data?}\label{sec:data-structure}

\boxx{Data are values of qualitative or quantitative variables, belonging to a set of items.\\
	
	\desx{
		\item[set of items:] Sometimes called the population; the set of objects you are interested in
		\item[variables:] A measurement or characteristic of an item.
		\item[qualitative:] Country of origin, sex, treatment
		\item[quantitative:] Height, weight, blood pressure
	}
}




Although the terms `\textbf{data}' and `\textbf{information}' are often used interchangeably, these terms have distinct meanings. Data is sometimes said to be transformed into information when it is viewed in context or in post-analysis. In academic treatments of the subject, however, data are simply units of information. [\dots]

Data is measured, collected and reported, and analyzed, whereupon it can be visualized using graphs, images or other analysis tools. Data as a general concept refers to the fact that some existing information or knowledge is represented or coded in some form suitable for better usage or processing. 





\section{Types of data}

\textbf{Raw data} (`unprocessed data') is a collection of numbers or characters before it has been `cleaned' and corrected by researchers. Raw data needs to be corrected to remove outliers or obvious instrument or data entry errors [\dots]`. Data processing commonly occurs by stages, and the `processed data' from one stage may be considered the `raw data' of the next stage.\\
\textbf{Field data} is raw data that is collected in an uncontrolled environment.\\
\textbf{Experimental data} is data that is generated within the context of a scientific investigation by observation and recording.\\
\textbf{Structured data:} Data is stored, processed, and manipulated in a traditional relational
database management system (e.g., temperature)\\
\textbf{Un-structured data} is commonly generated from human activities and does not
fit into a structured database format (e.g., text)\\
\textbf{Semi-structured data} does not fit into a structured database system, but is
nonetheless structured by tags that are useful for creating a form of order and
hierarchy in the data\\
\textbf{Big data} \textit{see below}\\
\textbf{Dark data} \textit{see below}


\pbn
\subsection{Structured vs. semi-structured data}
Business Intelligency requires analysts to deal with both structured and semi-structured data. The term semi-structured data is used for all data that does not fit neatly into relational or flat files, which is called structured data.
We use the term semi-structured (rather than the more common unstructured) to recognize that most data has some structure to it. A survey indicated that 60\% of CIOs and CTOs consider semi-structured data as critical for improving operations and creating new business opportunities \citep{Blumberg2003Problem}.

\begin{figure}[H]
	\begin{center}
		\includegraphics[width=.75\linewidth]{../../../pic/stvsus}
	\end{center}
	\caption{Structured vs. semi-structured data}
\end{figure}

\boxx{\subsubsection*{Executive at Fortune 500 telecommunciations provider}\textit{``We have between 50,000 and 100,000 conversations with our customers daily,
		and I don't know what was discussed. I can see only the end point -- for example,
		they changed their calling plan. I'm blind to the content of the conversations.''} \citep[see][]{Blumberg2003Problem}}





%\section{Online Survey}
%Go to\\ \url{https://pingo.coactum.de/455163}
%
%Give 3 examples for semi-structured data.
%}


\subsection{Semi-structured data: Examples}
For example, e-mail is divided into messages and messages are accumulated into file folders. 

Business processes, Chats, E-mails, Graphics, Image files, Letters, Marketing material, Memos, Movies, News items, Phone, conversations, Presentations, Reports, Research, Spreadsheet files, 
User group files, Video files, Web pages, White papers, Word processing text

\pbn
\exex{Data Mining}{
	\itex{
		\item Watch: \url{https://youtu.be/EH3bp5335IU}
		\item Read the Wikipedia page of `Data Mining'
		%\item Make slides that should introduce Data Minining to a student of IBM at the first semester in three minutes.
}}




\pbn
\subsection{Big data}
Big data is a field that treats ways to analyze, systematically extract information from, or otherwise deal with data sets that are too large or complex to be dealt with by traditional data-processing application software. [\dots]
%Data with many cases (rows) offer greater statistical power, while data with higher complexity (more attributes or columns) may lead to a higher false discovery rate. 
Big data challenges include capturing data, data storage, data analysis, search, sharing, transfer, visualization, querying, updating, information privacy and data source. Big data was originally associated with three key concepts: volume, variety, and velocity. When we handle big data, we may not sample but simply observe and track what happens. Therefore, big data often includes data with sizes that exceed the capacity of traditional software to process within an acceptable time and value. 

Current usage of the term big data tends to refer to the use of predictive analytics, user behavior analytics, or certain other advanced data analytics methods that extract value from data, and seldom to a particular size of data set. [\dots]
%``There is little doubt that the quantities of data now available are indeed large, but that's not the most relevant characteristic of this new data ecosystem.'' 
Analysis of data sets can find new correlations to ``spot business trends, prevent diseases, combat crime and so on.'' Scientists, business executives, practitioners of medicine, advertising and governments alike regularly meet difficulties with large data-sets in areas including Internet searches, fintech, urban informatics, and business informatics. Scientists encounter limitations in e-Science work, including meteorology, genomics, connectomics, complex physics simulations, biology and environmental research. \textbf{(Wikipedia)}


\pbn
%\subsubsection{Big data characteristics}
\begin{figure}[H]
	\begin{center}
		\includegraphics[width=0.6\linewidth]{../../../pic/Big_Data}
	\end{center}
	\caption{Big data characteristics}
\end{figure}

%Other mentioned data characteristics are:\\ Veracity, Exhaustive, Fine-grained and uniquely lexical, Relational, Extensional, Scalability, Value, Variability 



\pbn
\paragraph{Volume:} The amount of data matters. With big data, you’ll have to process high volumes of low-density, unstructured data. This can be data of unknown value, such as Twitter data feeds, clickstreams on a webpage or a mobile app, or sensor-enabled equipment.
\paragraph{Velocity:} The speed at which the data is generated and processed to meet the demands and challenges.  Normally, the highest velocity of data streams directly into memory versus being written to disk. Some internet-enabled smart products operate in real time or near real time and will require real-time evaluation and action
\paragraph{Variety:} Variety refers to the many types of data that are available. Traditional data types were structured and fit neatly in a relational
database. With the rise of big data, data comes in new unstructured data types. Unstructured and semistructured data types, such as
text, audio, and video, require additional preprocessing to derive meaning and support metadata


\includegraphics[width=.9\paperwidth]{$HOME/Dropbox/hsf/pic/big4}

\includegraphics[width=.6\paperwidth]{$HOME/Dropbox/hsf/pic/bd1a}

%\includegraphics[width=.6\paperwidth]{$HOME/Dropbox/hsf/pic/bd1b}

\pbn
\subsection{Dark data}
Dark data is data that is collected through various computer network operations but is not used in any way to gain insight or make decisions. An organization's ability to collect data may exceed its capacity with which to analyze the data. In some cases, the company is not even aware that the data is being collected. Approximately 90 percent of the data generated by sensors and analog-to-digital converters is never used.




\pbn
\section{What is data analytics?}

\boxx{The data is the second most important thing
	\itex{
		\item The most important thing in data science is the question
		\item The second most important is the data
		\item Often the data will limit or enable the questions
		\item But having data can't save you if you don't have a question
}}

\pbn
\boxx{\textbf{Data science} is an inter-disciplinary field that uses scientific methods, processes, algorithms and systems to extract knowledge and insights from many structural and unstructured data. Data science is related to data mining and big data.
	
	Data science is a ``concept to unify statistics, data analysis, machine learning and their related methods'' in order to ``understand and analyze actual phenomena'' with data. It employs techniques and theories drawn from many fields within the context of mathematics, statistics, computer science, and information science. 
}

\subsubsection*{Data analytics is what data scientists do}

\itex{
	\item Define the question
	\item Define the ideal data set
	\item Determine what data you can access
	\item Obtain the data
	\item Clean the data
	\item Exploratory data analysis
	\item Statistical prediction/modeling
	\item Interpret results
	\item Challenge results
	\item Synthesize/write up results
	\item Create reproducible code
	\item Distribute results to other people
}

\begin{figure}\centering
	\includegraphics[width=.6\textwidth]{$HOME/Dropbox/hsf/pic/data_scientist_sexy} 
	\caption{The sexiest job of the 21st century}
	\note{Source: \cite{Davenport2012Data}}
\end{figure}




\exex{What is data analysis}{
	Open the Wikipedia page of `Data Analysis'.
	\itex{\item Therein, the process of data analysis is described in eight steps. Read and try to memorize these steps.
		\item Stephen Few described eight types of quantitative messages that users may attempt to understand or communicate from a set of data and the associated graphs used to help communicate the message. Read and try to memorize these steps.
		\item The work of a data analyst contains many different tasks. Some of these tasks are described in the article. Which of these are new to you?}
}





\subsubsection*{Data analysis...}
\begin{minipage}{0.25\textwidth}	
	\includegraphics[width=.95\textwidth]{../../../pic/moneyball}
\end{minipage}
\begin{minipage}{0.75\textwidth}
	... is a process of inspecting, cleansing, transforming and modeling data with the goal of discovering useful information, informing conclusion and supporting decision-making. Data analysis has multiple facets and approaches, encompassing diverse techniques under a variety of names, and is used in different business, science, and social science domains. In today's business world, data analysis plays a role in making decisions more \textit{scientific} and helping businesses operate more effectively.
	
	\itex{\item	Please watch: \url{https://youtu.be/J36ZfXBsGjs}
		\item Sport Economics\footnote{It covers both the ways in which economists can study the distinctive institutions of sports, and the ways in which sports can allow economists to research many topics, including discrimination and antitrust law. } is a well-accepted discipline, see: \url{https://journals.sagepub.com/home/jse}	}
\end{minipage}




\section{Machine learning, AI, automated decision-making}


While \textbf{machine learning} (ML) is based on the idea that machines should be able to learn and adapt through experience, artificial intelligence (AI) refers to a broader idea where machines can execute tasks \textit{smartly}. AI applies ML techniques to solve actual problems and to automate decision making.


\subsection{Machine learning\dots}
\dots  is the study of computer algorithms that improve automatically through experience. In particular,  machine learning is a form of artificial intelligence (AI) as it provides machines and systems to automatically learn and improve from experience. Machine learning algorithms build a mathematical model based on sample data, known as `training data', in order to make predictions or decisions without being explicitly programmed to do so.
\begin{description}
	\item[\dots  for making predictions:] if you want a model to determine future
	trends; machine learning algorithms are the best bet. This falls under the paradigm of
	supervised learning. It is called supervised because you already have the data based on
	which you can train your machines (for example, a fraud detection model can be
	trained using a historical record of fraudulent purchases).\\
	
	\item[\dots for pattern discovery:] If you don’t have the parameters based on which you can make
	predictions, then you need to find out the hidden patterns within the dataset to be
	able to make meaningful predictions. This is nothing but the unsupervised model as
	you don’t have any predefined labels for grouping. The most common algorithm used
	for pattern discovery is Clustering.
\end{description}


%\section{Automated Decision-Making}

\subsection{What is automated decision-making?}
Automated decision-making is the process of making a decision by automated means without any human involvement. These decisions can be based on factual data, as well as on digitally created profiles or inferred data. Examples of this include:	
\itex{\item 	an online decision to award a loan; and
	\item an aptitude test used for recruitment which uses pre-programmed algorithms and criteria.
}
Automated decision-making often involves \textbf{profiling}, but it does not have to.

\boxb{
	\citet{Demetis2018When}:
	``Another well-known example comes from
	Amazon. The vast majority of prices are defined by
	algorithms in so far as Amazon vendors ``use algorithmic pricing to ensure that they can
	automatically change their product prices based on a
	competitor''' [39], with the result that vendors are
	being forced to engage in this practice for fear of
	losing out to the competition. Meanwhile, the
	algorithmic interactions between vendors carry the
	possibility
	of
	developing
	unpredictable
	consequences. Such algorithmic pricing on Amazon
	can be found in the example of the book entitled The
	Making of a Fly by evolutionary biologist Peter
	Lawrence. This book came to be priced at
	\textdollar23,698,655.93 (plus \textdollar3.99 shipping) as two sellers
	were using algorithms to adjust the price of the book
	in response to one another. It took 10 days for
	humans to notice and intervene to bring back the
	prices to normal levels [43]; ironically, ``normal
	levels'' merely indicated a temporary human decision
	that would allow the continuation of algorithmic
	pricing.''
}

%Please read section 2 of 
%\url{https://core.ac.uk/download/pdf/77240158.pdf}

\pbn
\subsection{What is profiling?}
Profiling analyzes aspects of an individual’s personality, behavior, interests and habits to make predictions or decisions about them.
In particular, profiling' means any form of automated processing of personal data consisting of the use of personal data to evaluate certain personal aspects relating to a natural person, in particular to analyze or predict aspects concerning that natural person's performance at work, economic situation, health, personal preferences, interests, reliability, behavior, location or movements.


\boxx{Watch: \tvbig\url{https://youtu.be/7-MNbzv8lAA}
	%\begin{center}
	\includegraphics[width=0.3\linewidth]{../../../pic/jdm/profile}
	%\end{center}
}

\boxb{\textbf{You are carrying out profiling if you:}
	\itex{\item collect and analyse personal data on a large scale, using algorithms, AI or machine-learning;
		\item identify associations to build links between different behaviours and attributes;
		\item create profiles that you apply to individuals; or
		\item predict individuals’ behaviour based on their assigned profiles.}}

%\pbn\paragraph{How does profiling work?}
Organizations obtain personal information about individuals from a variety of different sources. Internet searches, buying habits, lifestyle and behavior data gathered from mobile phones, social networks, video surveillance systems and the Internet of Things are examples of the types of data organizations might collect.

They analyze this information to classify people into different groups or sectors. This analysis identifies correlations between different behaviors and characteristics to create profiles for individuals. This profile will be new personal data about that individual.

\boxb{\textbf{Organizations use profiling to}
	\itex{\item 
		find something out about individuals’ preferences;
		\item predict their behavior; and/or
		\item make decisions about them.}}

Profiling can use \textbf{algorithms}. An algorithm is a sequence of instructions or set of rules designed to complete a task or solve a problem. Profiling uses algorithms to find correlations between separate datasets. These algorithms can then be used to make a wide range of decisions, for example to predict behavior or to control access to a service. Artificial intelligence (AI) systems and machine learning are increasingly used to create and apply algorithms. 
%For more information about algorithms, AI and machine-learning, big data, artificial intelligence, machine learning and data protection see the expert talks.

Although many people think of marketing as being the most common reason for profiling, this is not the only application.


\paragraph{What are the benefits of profiling and automated decision-making?}
Profiling and automated decision making can be very useful for organisations and also benefit individuals in many sectors, including healthcare, education, financial services and marketing. They can lead to quicker and more consistent decisions, particularly in cases where a very large volume of data needs to be analysed and decisions made very quickly.      


\paragraph{Examples}
Profiling is used in some medical treatments, by applying machine learning to predict patients’ health or the likelihood of a treatment being successful for a particular patient based on certain group characteristics.

Less obvious forms of profiling involve drawing inferences from apparently unrelated aspects of individuals’ behavior.

Using social media posts to analyze the personalities of car drivers by using an algorithm to analyze words and phrases which suggest ‘safe’ and ‘unsafe’ driving in order to assign a risk level to an individual and set their insurance premium accordingly.


\pbn
\paragraph{What are the risks?}

Although these techniques can be useful, there are potential risks:
\itex{\item 
	Profiling is often invisible to individuals.
	\item People might not expect their personal information to be used in this way.
	\item People might not understand how the process works or how it can affect them.
	\item The decisions taken may lead to significant adverse effects for some people.}

Just because analysis of the data finds a correlation doesn’t mean that this is significant. As the process can only make an assumption about someone’s behaviour or characteristics, there will always be a margin of error and a balancing exercise is needed to weigh up the risks of using the results. 

\pbn
\subsection{Industry 4.0}
Smart industry or \textit{INDUSTRIE 4.0} refers to the technological evolution from embedded systems to cyber-physical system. `INDUSTRIE 4.0 represents the coming fourth industrial revolution on the way to an Internet of Things, Data and Services. Decentralized intelligence helps create intelligent object networking and independent process management, with the interaction of the real and virtual worlds representing a crucial new aspect of the manufacturing and production process.



\begin{center}
	
	\includegraphics[width=0.6\linewidth]{../../../pic/jdm/i4dev}
\end{center}

\begin{center}
	
	\includegraphics[width=0.7\linewidth]{../../../pic/jdm/i4}
\end{center}


%
%\begin{center}
%	
%	\includegraphics[width=0.75\linewidth]{../../../pic/jdm/ii4}
%\end{center}








