\chapter{Git and GitHub}\label{ch:git}

\section{Git: Version control system}
\itex{
\item Here you find always up to data reference manuals, books, videos and links that help you to learn, understand, and apply git: \url{https://git-scm.com/doc}
\item \cite{Chacon2022Pro} is a detailed book (freely available) that answers all your questions about Git (and github): \url{https://git-scm.com/book/en/v2}
\item Here you find a cheatsheet for git: \url{https://training.github.com/}
}

\section{GitHub: platform to share and collaborate}
\url{GitHub.com} is an online platform ``where the world builds software''. It hosts files for software development and offers the distributed version control functionality of Git, plus its own features. It provides access control and several collaboration features such as bug tracking, feature requests, task management, continuous integration, and wikis for every project.

It is commonly used to host open-source projects. As of May 2022, GitHub reports having over 83 million developers, more than 4 million organizations, and more than 200 million repositories. It is currently the largest source code host and I guess that will not change in the future. 

\itex{
	\item A quick introduction can be found here: \url{https://docs.github.com/en/get-started/quickstart/hello-world}
	\item For a quick start I highly recommend to make your first contribution following this tutorial: \url{https://github.com/firstcontributions/first-contributions}
}

\section{Start contributing}

If you want to contribute to some other projects, I recommend to learn how to do it by following the instructions of this repository: \url{https://github.com/firstcontributions/first-contributions}

