\pbn
\section{Notes on the international flows of goods and capital}

\textbf{Net exports} are the value of domestic goods and services sold abroad minus the value of foreign
goods and services sold domestically. \textbf{Net capital outflow} is the acquisition of foreign assets by
domestic residents minus the acquisition of domestic assets by foreigners. Because every
international transaction involves an exchange of an asset for a good or service, an economy’s net
capital outflow always equals its net exports. 
An economy's saving can be used to finance investment at home or buy assets abroad. Thus,
national saving equals domestic investment plus net capital outflow.


\pbn
Definition of
\itex{
	\item  \textbf{exports}: goods and services that are produced domestically
	and sold abroad.
	\item \textbf{imports}: goods and services that are produced abroad and
	sold domestically.
	\item \textbf{net exports}: the value of a nation's exports minus the value
	of its imports, also called the trade balance.
	\item  \textbf{trade balance}: the value of a nation's exports minus the
	value of its imports, also called net exports.
	\item  \textbf{trade surplus}: an excess of exports over imports.
	\item  \textbf{trade deficit}: an excess of imports over exports.
	\item  \textbf{balanced trade}: a situation in which exports equal imports.
	\item \textbf{net capital outflow}: the purchase of foreign assets by
	domestic residents minus the purchase of domestic assets by
	foreigners.}

\pbn
The flow of capital abroad can take two forms:
\abcx{\item  Foreign direct investment occurs when a capital investment is owned and operated by a foreign entity. \item Foreign portfolio investment involves an investment that is financed with foreign money but operated by domestic residents.}

\pbn
\subsubsection*{Examples}
Boeing sells some airplanes to a Japanese airline.
\abcx{
	\item Boeing gives planes to the Japanese firm, and the Japanese firm gives
	yen to Boeing. Exports have increased (which raises net exports) and
	the United States has acquired some foreign assets in terms of yen
	(which raises net capital outflow).
	\item  Or Boeing may exchange its yen for dollars with another entity that
	wants yen. Suppose an American mutual funds wants to buy some stock
	in a Japanese company. In this case, Boeing’s net export of planes
	equals the mutual fund’s net capital outflow in stock.
	\item  Or Boeing may exchange its yen with an American firm that wants to
	buy some good or service from a Japanese company. In this case, the
	imports will exactly offset the exports, so net exports is un
}

\pbn
Every international transaction involves exchange. When a seller country
transfers a good or service to a buyer country, the buyer country gives up some
asset to pay for the good or service.
Thus, the net value of the goods and services sold by a country (net exports)
must equal the net value of the assets acquired (net capital outflow).



\pbn
\section{Balance of payments}\label{sec:balance-of-payments}
%

\exex{Germany's Net Exports}{
	Discuss the pros and cons of Germany's net export surplus.
}

\solx{Germany's Net Exports}{
	Please watch: \tv \url{https://youtu.be/OE-dXYeFmtg}.
	
	\begin{minipage}[t]{0.5\textwidth}	
		\begin{center}
			\includegraphics[width=.8\linewidth]{$HOME/Dropbox/hsf/pic/ie/nex-ger_png}
			\captionof{figure}{Germany's Net Export}\label{fig:nex-ger}\bigskip
		\end{center}
		
	\end{minipage}
	\begin{minipage}[t]{0.5\textwidth}
		\begin{center}
			\includegraphics[width=.8\linewidth]{$HOME/Dropbox/hsf/pic/ie/cex-ger_png}
			\captionof{figure}{Foreign Assets of Germans}\label{fig:cex-ger}\bigskip
		\end{center}
		
	\end{minipage}
}

%\titx{Motivation:  Das Yin und das Yang der Leistungsbilanz}{Read the following article from Jens Südekum und Gabriel Felbermayr, published in the Frankfurter Allgemeine Zeitung (03.04.2017, Nr. 79, S. 16)
%	
%	\textbf{Das Yin und das Yang der Leistungsbilanz}
%	
%	\textit{Warum der sehr hohe deutsche Leistungsbilanzüberschuss kein Grund zur reinen Zufriedenheit ist.}
%	
%	Unausgeglichene Handelsbilanzen erregen die Gemüter. Dafür hat nicht zuletzt der Vorwurf von Donald Trumps Wirtschaftsberater Peter Navarro gesorgt, Deutschland manipuliere den Euro, um sich auf Kosten anderer zu bereichern. Auch wenn dieser Vorwurf absurd ist, denn der Außenwert des Euros wird nicht in Berlin entschieden, bleibt hierzulande doch ein mulmiges Gefühl zurück. Steckt vielleicht ein Körnchen Wahrheit darin?
%	
%	Immerhin wurde Deutschland kürzlich wieder zum Exportweltmeister gekürt, und unser Leistungsbilanzüberschuss ist mit knapp 9 Prozent des Bruttoinlandsprodukts größer als je zuvor. Der Internationale Währungsfonds, die EU-Kommission und viele Ökonomen weltweit fordern Korrekturen. Müssen wir uns also Sorgen machen, dass der Überschuss "zu hoch" ist? Und was könnten wir tun, um ihn zu reduzieren? Die Krux mit der Leistungsbilanz ist, dass man sie immer von zwei Seiten betrachten muss. Diese beiden gehören unzertrennlich zusammen wie das Yin und das Yang aus der chinesischen Philosophie. Schaut man bloß isoliert auf eine Seite, dann entstehen schnell Fehlschlüsse.
%	
%	Das "Yin" der Leistungsbilanz geht so: Deutschland hat 2016 Waren im Wert von 1,2 Billionen Euro exportiert. Die Wareneinfuhr betrug 920 Milliarden Euro, also 280 Milliarden Euro weniger. Auf der Dienstleistungsbilanz haben wir zwar ein kleines Defizit, vor allem wegen vieler Auslandsurlaube, die saldenmechanisch Importe sind. Zählt man alles zusammen, kommt Deutschland zu einem Überschuss in der Leistungsbilanz von 261 Milliarden Euro. Viele sehen das als Beweis für die enorme Stärke der Wirtschaft. Es gibt hier eben exzellente Unternehmen mit Produkten von höchster Qualität. Die Deutschen sind erfolgreicher als andere, also wollen alle mehr von uns kaufen als umgekehrt. Dadurch entstehen hier Tausende zusätzlicher Arbeitsplätze, um die Auslandsnachfrage zu befriedigen, während Amerika zu Lasten der einfachen Arbeiter mit Importen überflutet werde.
%	
%	Diese hemdsärmelige Sicht, wonach Exportüberschüsse tugendhaft sind, springt volkswirtschaftlich allerdings viel zu kurz. Das wird deutlich, wenn man das "Yang" der Leistungsbilanz betrachtet, die Finanzierungsseite. Hiernach hat die deutsche Volkswirtschaft allein im vergangenen Jahr 261 Milliarden Euro weniger konsumiert oder im Inland investiert, als möglich gewesen wäre. Dieses Geld ist natürlich nicht einfach weg. Es wurde gespart und im Ausland angelegt. Doch man muss es sich auf der Zunge zergehen lassen: Deutschland wird oft als sicherer Hafen bezeichnet. Aber im Saldo hat mehr Kapital das Land verlassen, als hineingeflossen ist. So ist das deutsche Netto-Auslandsvermögen auf stolze 1,8 Billionen Euro angewachsen, erzielt laut Bundesbank aber miserable Renditen. Plötzlich hört sich ein Leistungsbilanzüberschuss gar nicht mehr so toll an: Statt die Früchte der eigenen Arbeit im Inland zu genießen, leihen wir lieber dem Ausland ständig mehr Geld und hoffen, dass wir es irgendwann wiedersehen. Ganz anders die Vereinigten Staaten: Amerika mit seinem "Dollarprivileg" darf jedes Jahr mehr konsumieren, als es produziert, weil Deutschland, Japan und China auf Pump exportieren. Ob Trump bewusst ist, dass dies vorbei wäre, wenn er das amerikanische Defizit abbaut?
%	
%	Nun gibt es handfeste Gründe, warum Deutschland einen Leistungsbilanzüberschuss und damit einen Kapitalexport aufweist. Der wichtigste ist die Demographie: Wir sind eine alternde Gesellschaft und sorgen uns um die Altersvorsorge. Gleichzeitig ist Deutschland eine reiche Volkswirtschaft, während viele Schwellenländer für ihr Wachstum noch auf Kapitalzuströme angewiesen sind. Insofern ist es ganz natürlich, wenn Deutschland viel spart und diversifiziert im Ausland anlegt. Aber dieses Argument gilt für viele Länder, die eine ähnliche Einkommens- und Altersstruktur wie Deutschland aufweisen, aber längst nicht so positive Handelsbilanzsalden. Italien oder Spanien etwa, oder das noch schneller alternde Japan.
%	
%	Der deutsche Überschuss hat schon etwas Spezielles, und das liegt am Euro. Innerhalb der Währungsunion mangelt es mehr denn je an realer Konvergenz. Während bei uns nahezu Vollbeschäftigung herrscht, hält sich die Jugendarbeitslosigkeit in Südeuropa auf bedrohlichem Niveau. Die (Kern-)Inflation ist unter der 2-Prozent-Marke. Somit bleibt die Europäische Zentralbank vorerst bei ihrer ultralockeren Geldpolitik, die den Euro billig hält. Diese Politik ist eigentlich nicht im deutschen Interesse. Sie wird betrieben, um diverse Banken und damit den Euro am Leben zu erhalten. Ein Nebeneffekt ist, dass deutsche Unternehmen wegen des "zu billigen" Wechselkurses und der bescheidenen vergangenen Lohnrunden derzeit besonders wettbewerbsfähig sind.
%	
%	Aber was soll Deutschland dagegen tun? Den starken Unternehmen das Exportieren verbieten? Natürlich nicht! Wir sollten vielmehr überlegen, warum wir die enormen Exporterlöse nicht vermehrt für den Konsum schicker Importgüter oder für inländische Investitionen in Maschinen, Schulen oder Breitbandnetze verwenden. Global lässt sich das schwerlich steuern, denn Leistungsbilanzen ergeben sich im Aggregat aus Millionen von Einzelentscheidungen. Aber wenn es systematische Verzerrungen bei diesen Entscheidungen gibt, dann gilt es dort anzusetzen.
%	
%	Wichtig sind konsequente Reformen im Dienstleistungssektor. Hier wird die Zukunft der Arbeit liegen, dafür sorgt schon die Digitalisierung. Aber wie die OECD jedes Jahr aufs Neue feststellt, sind viele Branchen in Deutschland stärker reguliert als anderswo. Dies hat hohe Preise und geringe Investitionen zur Folge. Das Schattendasein von Uber \& Co im deutschen Markt ist dabei nur ein Beispiel. Auch die Unternehmensbesteuerung gehört auf den Prüfstand. Das derzeitige System macht es für multinationale Konzerne sehr attraktiv, erzielte Gewinne als Rücklage im Ausland zu horten, statt im Inland auszuschütten. Eine Senkung der Mehrwertsteuer würde die (Import-)Nachfrage anregen, was nebenbei auch Südeuropa hülfe. Sie würde zudem die Schwächsten hierzulande entlasten, die keine Einkommen-, wohl aber Mehrwertsteuer zahlen. Und schließlich müssen Lücken bei der öffentlichen Infrastruktur geschlossen werden, deren Ausbau seit längerer Zeit stagniert. Bei keiner dieser Maßnahmen verschenkt Deutschland Geld oder verschuldet sich übermäßig. Im Gegenteil: Der Weg zu einer stärker ausgeglichenen Leistungsbilanz über eine höhere Binnennachfrage könnte geradezu freudestiftend sein, denn Konsum macht Spaß. Und eine stärkere Rückbesinnung auf sich selbst ist auch im Sinne von Yin und Yang.
%}
%
%


\subsection{The balance of payments must be balanced!}
A country's \textit{Balance of Payments} account (Zahlungsbilanz) records the payments and receipts of its residents in their transactions with residents of other countries. 
If all transactions are included, the payments and receipts of each country are, and must be, \textit{balanced}. Any apparent inequality simply leaves one country acquiring assets in the others. 
The \textit{Balance of Payments} account consists of two primary components:
\boxx{
	\begin{enumerate}
		\item The \textbf{Current Account} (Leistungsbilanz) which measures a country's trade balance plus the effects of net income and direct payments. In particular, it consists of four components, that are, trade, net income, direct transfers of capital, and asset income.
		\item The \textbf{Capital Account} (Kapitalbilanz) reflects the net change in ownership of national assets:
		\begin{align*}
			\textnormal{Capital account} = & \textnormal{ Change in foreign ownership of domestic assets}\\ &- \textnormal{ Change in domestic ownership of foreign assets}
		\end{align*}
		%	A deficit means a country is increasing its ownership of foreign assets, whereas a surplus means foreign countries become owner of domestic assets, for example through foreign direct investments.
		%	\item Official Reserves (Währungsreserven)
	\end{enumerate}
}
Ignoring statistical effects, these two subaccounts must sum to zero. Figure \ref{fig:balance} shows that this appears to be the case for the United States. 
%See also on page \pageref{pag:balance} Germany's balance of payments account.

\begin{center}
	\includegraphics[width=.85\linewidth]{$HOME/Dropbox/hsf/pic/makro/balance.pdf}
	\captionof{figure}{U.S. Balance of Payments}\label{fig:balance}\bigskip
\end{center}

\pbn
Although the totals of payments and receipts are necessarily equal, certain types of transactions give rise to imbalances-surpluses of payments or receipts called deficits or surpluses. Thus, deficits or surpluses can occur in the following areas: Trade in goods (commodities), trade in services, foreign investment income, unilateral transfers (foreign aid), private investment, flows of gold and money between central banks and treasuries, or a combination of these or other international transactions. It should be made clear, however, that these surpluses and deficits must add up to zero, i.e., payments must balance (double-entry bookkeeping!).

For example, if the Americans buy cars from Germany and have no other transactions with Germany, the Germans must end up holding dollars, which they can hold in the form of bank deposits in the United States or in some other U.S. facility. Americans' payments to Germany for automobiles are offset by Germany's payments to U.S. persons and institutions, including banks, for the purchase of dollar assets. In other words, Germany sold automobiles to the United States, and the United States sold dollars or dollar-denominated assets to Germany.
Thus, Germany runs a trade surplus meaning its \textit{Trade Balance} (Handelsbilanz) is positive and so is the \textit{Current Account} (Leistungsbilanz), which includes the \textit{Trade Balance}. 
However, Germany must also have a deficit in the capital account. In other words, more money is flowing out than coming in. 

\pbn
\subsection{A formal representation}
In what follows, I provide a simplified overview of how the world trading system works. I do not go into the pros and cons of running a trade surplus or deficit. This is a topic in itself. However, I do try to provide an understanding of the determinants of current account deficits and surpluses. 

\subsubsection*{Closed Economy}
In a closed economy, there are three main agents: households, firms, and the government. If we denote $C$ as the consumption of goods and services by households (food, housing, entertainment, ...), $G$ as the purchases by the government (infrastructure, social services, military spending, education, ...), and $I$ as the level of investment by firms (machinery, buildings, research and development, ...), then we can write total output $Y$ as 

\begin{equation}
	Y=C+I+G.\label{eq:fundamental_close}
\end{equation}
If we define national savings, $S$, as the share of output not spend on household consumption or government purchases, \[S\equiv Y-C-G,\]
then the investments, $I$, must be equal to the savings in a closed (!) economy:
\[S=I,\] 
which can easily been shown by rearranging equation (\ref{eq:fundamental_close}) as follows
\begin{align*}
	I&=\underbrace{Y-C-G}_{\equiv S}.
\end{align*}

\subsubsection*{Open Economy}
In an open economy, household consumption, government purchases, and firms investments may not be produced in the home economy but may be imported from abroad. Similarly, the home production may be exported to foreign consumers, firms, or governments. Thus, an economy can import and export goods.
Denoting imports by $IM$ and exports by $EX$, we can re-write equation (\ref{eq:fundamental_close}) as follows:
\begin{align}
	Y&=C+I+G+EX-IM\label{eq:open1}
\end{align}
If an economy exports more than  it imports, $EX>IM$, it has a trade surplus. If  an economy exports less than it imports, $EX<IM$, it has a trade deficit. The difference between exports and imports can be called net-exports, $NEX$. That may represent the current account balance (at least, if we abstract from other income and financial transactions). Rearranging equation (\ref{eq:open1}) we can write
\begin{align}
	% Y&=C+I+G+EX-IM\\
	\Leftrightarrow	\underbrace{Y-C-G}_{\equiv S}-I&=\underbrace{EX-IM}_{\equiv NEX}\\
	\Leftrightarrow S-I=NEX.
\end{align}
When $I=S$, the share of output not spend on household consumption or government purchases equals the investments and, in turn, the economy has zero net-exports. If an economy, however, has a trade surplus, $NEX>0$, such as Germany in the last decade, the savings exceeds the investments. That means, the country produces more than it is spending on goods and services. Thus, the savings that are not used domestically, $S-I$, are invested abroad. That means, the country with the trade surplus is acting as a lender to or investor in the rest of the world.
The difference $S-I$ hence can be denoted as the net capital outflow, $NCO$:
\begin{align}
	\underbrace{S-I}_{\equiv NCO}&=NEX\\
	NCO&=NEX
\end{align}

\heux{Net Exports Must Be Equal to Net Capital Outflow}{ The accounting identities above simple state that there is a \textit{balance of payments}. The Balance of Payment accounts are based on double-entry bookkeeping and hence the annual  account has to be balanced. If an economy has a current account trade deficit (surplus) is offset one-to-one by a capital account surplus (deficit) to assure a balance of payments. 
	In other words, if an economy wants to import more goods than it produces, it must attract foreign capital being invested at home. }

What would happen if the United States, for example, could not attract capital flows from the rest of the world to finance their trade deficit? 
This would mean that American consumers buy foreign goods with US-Dollars and more US-Dollars are flowing out of the country than coming in. In turn, the supply of US-Dollars is greater than the demand and the US-Dollar would fall in value. This depreciation of the US-Dollar would make US-exports cheaper and imports more expensive and hence the current account deficit would be reduced. However, so far the export deficit of the United States is kind of stable and the US-Dollar does not depreciate substantially. 
This is probably the reason why President Trump claimed other countries to \textit{manipulate} their currencies, see \autoref{fig:manipul}. 
As Trump thinks a trade deficit is bad for the United States, he would like to have a weak dollar\footnote{I guess he would never put it that way.} and low interest rates. 
A weak dollar makes American products cheap for the rest of the world and has positive effects on exports and negative on imports (see Heureka \ref{heu:Exchange rates and international trade}). A low interest rate in the United States would make the country less attractive for foreign capital investments ($I$ would become smaller), i.e., the net capital inflows would decrease and so would the \textit{Capital Account}'s surplus (and with it the Current Account deficit would become smaller).  
In concrete terms, he claims that in particular the Chinese government and the European Central Bank run policies that hold their currencies (Renminbi and Euro) cheap.
\bigskip
\begin{center}
	\includegraphics[width=.55\linewidth]{$HOME/Dropbox/hsf/pic/ie/manipul_pdf}
	\captionof{figure}{Trump worries about the U.S. trade deficit}	\note{Source: Twitter}\label{fig:manipul}
\end{center} 



%
%\includepdf[pages={4}]{$HOME/Dropbox/hsf/pic/ie/Zahlungsbilanz_pdf}\label{pag:balance}

%\exex{Explaining export surpluses (deficits)}{
%	\enxx{(a)}{
%		\item Discuss the determinants of a trade surplus and deficit, respectively.
%%		\item Assume the European Central Bank increases the interest rates. What are the implications for that on net-exports?
%		\item Discuss the pros and cons of a trade surplus and deficit, respectively.
%}}
%
%\solx{Explaining export surpluses (deficits)}{
%Please join the in-class discussion.}