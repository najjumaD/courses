
\chapter{\textcolor{cyan}{\{}swi\textcolor{cyan}{r}l\textcolor{cyan}{\}}-it}

%		\begin{center}
%		\includegraphics[width=.4\linewidth]{../../../pic/Rstudio/swirlr}
%	\end{center}
\boxx{
Required exercises:
\itex{
	\item Students should go through the following learning modules that are all part of my \textit{swirl-it} course:
	\itex{
		\item \textit{huber-intro-1}
		\item \textit{huber-intro-2}
		\item \textit{huber-data-1}
		\item \textit{huber-data-2}
		\item \textit{huber-data-3}
	}	
}
}

\section{What is \textcolor{cyan}{\{}swi\textcolor{cyan}{r}l\textcolor{cyan}{\}}?}

\texttt{swirl} teaches you \R programming and data science interactively, at your own pace, and right in the \R console!
Just follow the instructions on \websmall\url{https://swirlstats.com/students.html} and you will learn \R step by step within \Rstudio itself.

\section{A short introduction to \R and \Rstudio in two \textcolor{cyan}{\{}swi\textcolor{cyan}{r}l\textcolor{cyan}{\}} modules}\label{exe:swirl}
The \textit{swirl} \R package makes it fun and easy to learn \R programming and data science. If you are new to \R, have no fear. \textit{swirl} will walk you through each of the steps required to employ \Rstudio and \R for your purpose.

Open \Rstudio  and type in the console the following:

\begin{rblock1}
	install.packages("swirl")
	library("swirl")
	ls()
	rm(list=ls())
\end{rblock1}

The above lines of code do the following:

\begin{itemize}
	\item Install the \textit{swirl} package.
	\item Load the \textit{swirl} package.
	\item List the content of the environment.
	\item Remove everything from the environment.
	\item Start \textit{swirl}.
\end{itemize}

Now type in the Console the following:\footnote{If the course has failed to install, you can try to download the file \texttt{swirl-it.swc} from \url{https://github.com/hubchev/swirl-it} and install the course with \rtext{install\_course()}}
\begin{rblock1}
	install_course_github("hubchev", "swirl-it")
\end{rblock1}
With the first line, you install my \textit{swirl} course that is hosted on GitHub.

To start swirl and your learning experience type
\begin{rblock1}
	swirl()
\end{rblock1}

With \rtext{swirl()} you start \textit{swirl}. 
Please choose the course \textit{swirl-it} and the learning module \textit{huber-intro-1}.
You can exit \textit{swirl} at any time by typing \rtext{bye()} or by clicking the \rtext{Esc} on your keyboard. 

After you have successfully finished learning module \textit{huber-intro-1} please go ahead with the learning module \textit{huber-intro-2} that is also part of my swirl course \textit{swirl-it}.



\section{\textcolor{cyan}{\{}swi\textcolor{cyan}{r}l\textcolor{cyan}{\}} module on data analytical basics}

In my swirl modules \textit{huber-data-1}, \textit{huber-data-2}, and \textit{huber-data-3} I introduce some very basic statistical principles on how to analyse data. These modules are part of my \textit{swirl-it} course that can be installed as explained in \autoref{exe:swirl}.


\section{\textcolor{cyan}{\{}swi\textcolor{cyan}{r}l\textcolor{cyan}{\}} module on the \textit{tidyverse} package}
I compiled a short \textit{swirl} module to introduce the \textit{tidyverse} universe. This is a powerful collection of packages which I discuss in \autoref{ch:tidyverse}. The learning module is also part of my \textit{swirl-it} course.

\section{Other \textcolor{cyan}{\{}swi\textcolor{cyan}{r}l\textcolor{cyan}{\}} courses}
You can also install some other courses. You find a list of courses here \websmall\url{http://swirlstats.com/scn/index.html} or here \websmall\url{https://github.com/swirldev/swirl_courses}

I can recommend the following:
\begin{rblock1}
	swirl::install_course_github("swirldev", "R_Programming_E")
	swirl::install_course_github("matt-dray", "tidyswirl")
%	swirl::install_course("Exploratory Data Analysis")
	swirl::install_course("Getting and Cleaning Data")
	swirl::install_course_github("sysilviakim", "swirl-tidy")
	swirl::install_course("Regression Models")
\end{rblock1}



